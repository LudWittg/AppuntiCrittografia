\documentclass{article}
\usepackage[utf8]{inputenc}
\usepackage[italian]{babel}
\usepackage{amsmath}
\usepackage{amssymb}

\title{Distillato di Crittografia}
\author{Michele Sponsale }
\date{Luglio 2019}

\begin{document}

\maketitle

\section{Casualità e Complessità di Kolmogorov}

Sia $S_i$ l'$i$-esimo \emph{sistema di calcolo} in una certa enumerazione dei sistemi di calcolo\footnote{Tralasciamo la dimostrazione della numerabilità dei sistemi di calcolo Turing-equivalenti.}. Se $p$ \`e un programma (stringa) di lunghezza $|p|$ per calcolare il numero $h$ in $S_i$, indicato come $S_i(p)=h$, allora la \textbf{Complessita di Kolmogorov \emph{relativa}} a $S_i$ di $h$ $K_{S_i}(h)$ è
\[
K_{S_i}(h)=\min\{\left|x\right|:\, S_i(p)=h\}
\]
\emph{Facoltativo:}
Esistono dei sistemi di calcolo universali $S_U$ tali che possano emulare un programma $p$ per il calcolo di $h$ di qualunque $S_i$ dato il programma $<p,i>$ (concatenazione di stringhe) di lunghezza $\left|p\right|+\log i$\,,\footnote{La lunghezza di un numero naturale rappresentato in base $B$ è $\log_B i$.} ossia $S_U(p,i) = h$. 

La complessità di Kolmogorov su macchine universali o semplicemente \textbf{Complessità di Kolmogorov} di un numero $h$ rispetta, per ogni $i$,
\[
K(p)=K_{S_U}(p) \le K_{S_i}(p)+c_i
\]
dove $c_i$ è una costante che dipende solo da $i$.
Riassumendo meno formalmente
\fbox{
\parbox{\textwidth}{
 La complessità di Kolmogorov di un numero è la lunghezza del più corto programma (senza input\dots) che lo calcola.
    }
}

\subsection{Numero casuale}

Un numero $h$ è detto (\textit{algoritmicamente}) \textbf{casuale} se
\[
K(h)\ge \left|h\right|
\]
Statisticamente le proprietà di un numero che verifica la precedente sono le stesse di un numero che verifichi la seguente \textbf{definizione di numero \emph{casuale}}
\[
K(h) \ge \left|h\right| - \lceil\log_2 \left| h\right|\rceil
\]

\subsection{Cardinalità insieme numeri casuali - quanti ce ne sono}
%\subsubsection{Consigliato, ma facoltativo}

Fra i numeri di lunghezza (per esempio in bit) minore o uguale a $n$, la quantità dei numeri casuali è esponenziale alla lunghezza ed esistono di ogni lunghezza, come è possibile dimostrare con considerazioni sulla complessità di Kolmogorov.
Dunque, fino a un numero fissato, sono esponenzialmente più frequenti i numeri casuali che quelli non casuali.

Stabilire se un numero sia casuale o meno (nel senso di Kolmogorov) è indecidibile, tuttavia possiamo testare le proprietà statistiche di un numero affinché siano simili a quelle di un numero casuale.

\subsection{Test statistici di (pseudo)casualità}

Superare un test statistico di casualità è condizione necessaria (ma non sufficiente) perché un numero sia casuale (dal paragrafo precedente non possono esistere classicamente condizioni sufficienti).

\paragraph{Test del prossimo bit}

Un numero supera il \emph{test del prossimo bit} se il bit $k+1$ del numero non può essere previsto con probabilità maggiore di $\frac{1}{2}$ conoscendo i suoi precedenti $k$ bit. Questo è il test più forte.
%Ci sono $2^n$ sequenze binarie di $n$ bit, e $2^{n-\lceil\log_2 n\rceil}-1$ sequenze più corte di ${n-\lceil\log_2 n\rceil}$ (cioé di al più ${n-\lceil\log_2 n\rceil}-1$ bit). %Tra quest'ultime ci sono T sequenze dei programmi che generano numeri \emph{non-casuali} di valore al più $n$ (la loro lunghezza deve essere minore di $ \left|h\right| - \lceil\log_2 \left| h\right|\rceil$ per definizione), dunque %\[%T(n)\limits_{\text{cardinalità numeri non-casuali di bit }<n} %\le {n-\lceil\log_2 n\rceil}%\limits_{\text{cardinalità massima non-casuali}}%< 2^n\limits_{\text{cardinalità numeri di lunghezza} n}%\]
\hrule
\emph{Consigliato ma non necessario}

Questo test implica almeno altri 4 test statistici (nel senso che se $n$ supera il test del prossimo bit supera anche questi):
\paragraph{Test di frequenza}
Ogni elemento della sequenza che rappresenta il numero compare quanto ogni altro.
\paragraph{Poker test}
Data una lunghezza fissata $k$, un numero passa il \emph{poker test} se tutte le sottosequenze di lunghezza $k$ (i \emph{poker}) appaiono con la stessa frequenza.
\paragraph{Test di autocorrelazione}
Verifica il numero di elementi uguali ripetuti a distanze prefissate (ogni 5 elementi, ogni 3, etc.)
\paragraph{Run test}
Verifica se sottosequenze massimali di elementi identici\footnote{E.g., in $111019011111111$ le sottosequenze massimali sono $111,0,1,9$ e $11111111$ ma non $11111$.} seguano una distribuzione esponenziale negativa nella lunghezza.
\hrule

\section{Generatori di numeri pseudo-casuali}
Un \textbf{generatore di numeri pseudo-casuali} è un algoritmo o macchina fisica che genera numeri che superano alcuni test statistici di casualità. Nella pratica è utilizzato per generare una sequenza più o meno lunga di numeri casuali a partire da un singolo numero casuale (``amplificano'' la casualità). Il \emph{periodo} di un generatore è il minimo valore della variabili indipendente, chiamata in questo contesto \emph{seme}, dopo il quale l'immagine si ripete. 

Esempi di generatori di numeri pseudo-casuali sono i \textbf{generatori polinomiali congruenti} $P_n(x)\mod m$ dove $P_n(x)$
è un polinomio di grado $n$,
in cui tuttavia i coefficienti di $P_n(x)$
sono soggetti a vincoli affinché il generatore funzioni\footnote{Questi vincoli sulle funzioni modulari ricompaiono per esempio nel \emph{test di primalità di Miller-Rabin}.}, oppure il \textbf{generatore BBS} 
(\emph{Blum-Blum-Shub} dai nomi dei progettisti).

\section{Predicati hard-core e generatore BBS}
\subsection{Accenno su funzioni one-way}
Una funzione $f$ è detta one-way se \[
f\in PTIME \land f^{-1}\in EXPTIME
\]
(la funzione è calcolabile da algoritmi polinomiali ma la sua inversa solo da algoritmi esponenziali). Molto spesso a fini pratici per funzioni one-way si rilassa la condizione a $f\in P$ e $f^{-1}\in NPH$ o $f^{-1}\in NP$, e addirittura spesso si considerano per comodità pratica funzioni tali che si conoscano solo algoritmi $NP$, ma la cui appartenenza alla classe non è dimostrata (quindi possono essere scoperti algoritmi più efficienti).
Riassumendo informalmente\\
\\
\fbox{
\parbox{\textwidth}{
Le funzioni one-way sono funzioni facili da calcolare ma computazionalmente difficili da invertire.
    }
}

\subsection{Predicati hard-core}
Un predicato $b(x)$ (cioé una funzione che restituisce $0$ o $1$) è detto \textbf{hard-core} per una funzione $f$ se soddisfa la seguente definizione\\
\\
\fbox{
\parbox{\textwidth}
    {
Un predicato $b$ per una funzione $f$ è detto \textbf{hard-core} \textit{sse} $b(x)\in P$ e prevedere il valore di $\widetilde{b}(f(x))=b(f^{-1}(f(x)))$ con una probabilità maggiore di $\frac{1}{2}$ è un problema NP. 
    }
}\\
\\
Dunque se un predicato è facile da calcolare conoscendo $x$ ma è difficile da prevedere conoscendo $f(x)$, allora è \textit{hard-core} per quella funzione. In generale questi predicati esistono tipicamente per funzioni \emph{one-way}, perché ereditano la difficoltà di previsione dalla difficoltà di dover invertire $f$ per calcolarli.

Un tipico modo di utilizzare predicati \textit{hard-core} è calcolare iterativamente da $0$ a $N$ una sequenza di coppie $ f^i(x),b(f^i(x))$ ($f$ applicata a se stessa $i$-volte), che è facile da calcolare per definizione, e successivamente restituire come valori le coppie nell'ordine inverso da $N,N-1,\dots$ a $0$. Restituendo i valori in ordine inverso una sequenza di bit generata in questo modo supera per definizione il \textbf{test del prossimo bit}.

\subsection{Generatore BBS}
Prendo due numeri primi grandi $p$ e $q$, che soddisfino \textbf{entrambi}
\hrule
\begin{enumerate}
    %\item $n\leftarrow pq$
    %\item $j\in\{p,q\}$
    \item $j\mod 4 = 3$
    \item $ 2\lfloor{\frac{j}{4}}\rfloor+1$ è primo 
\end{enumerate}
\hrule

Allora, preso un numero $y$ a caso (\textit{seed} del \textbf{BBS}) uso la funzione one-way (perché invertirla è difficile)
\[
x_0\leftarrow y^2 \mod pq
\]\[
x_{i+1}\leftarrow x_i^2 \mod (pq)
\]
e ad ogni passo dell'algoritmo iterativo calcolo il predicato predicato hard-core
\[
b(x_i)=x_{i-1}\mod 2
\]
(che fa $0$ se $x_{i-1}$ calcolato al passo precedente è pari e $1$ se dispari).
Alla fine restituisco i valori del predicato di parità in ordine inverso, generando dunque un numero che supera il \textit{test del prossimo bit} poiché il predicato è hard-core.

\section{Test di primalità di Miller-Rabin}
Il test di Miller-Rabin determina se un numero è primo con una probabilità di errore piccola a piacere scelta al momento dell'esecuzione (algoritmo Monte Carlo). Si basa sulle proprietà statistiche dei numeri primi conseguenza del Lemma di Miller-Rabin, ed è un metodo iterativo.
\subsection{Lemma di Miller-Rabin}
Se $N$ è pari allora è sicuramente composto (non è un numero primo), altrimenti se è \textbf{dispari}
\[
N-1\text{ è pari}\to N-1=2^w z\text{ con z dispari}
\]
poiché ovviamente qualsiasi numero pari si può fattorizzare come una potenza di $2$ per un numero dispari.
\paragraph{Proprietà} Dato un qualunque $y|2\le y\le N-1$ allora se $N$ è primo valgono le due proprietà (condizioni necessarie):
\hrule
\begin{enumerate}
    \item[P1.] $MCD(N,y)=1$ 
    \item[P2.] $y^z\mod N=1 \lor \exists i(0\le i\ge w-1):y^{2^i z}\mod N =-1$
\end{enumerate}{}
\hrule
\subsubsection{Lemma} 

\fbox{
\parbox{\textwidth}
    {
Se $N$ è composto il numero di interi in $[2,N-1]$ che soddisfano entrambe le proprietà è minore di $N/4$.
    }
}
\subsection{Il test}
Se per un $y$ \textbf{SCELTO A CASO}\footnote{Scegliendo non a caso il ragionamento non funziona.} è falsificata P1 o P2 allora $N$ è composto, ma anche se non si trovasse un $y$ non potremo concludere la primalità di $N$, ma solo che la probabilità che $N$ sia composto è minore di $1/4$ (equivalentemente la probabilità che sia primo è maggiore di $3/4$). In questo caso chiameremo $y$ \textit{certificato} (\textit{probabilistico}) della primalità di $N$.

Definiamo la subroutine di utilità, dato un numero $N$ (dispari) e un (papabile) certificato di primalità $y$
\hrule Verifica($N$,$y$):\\
\textbf{if not} P1 or \textbf{not} P2:\\
\indent    \textbf{return composto}\\
\indent    \textbf{else return indefinito}
\hrule
che ci permette di descrivere l'algoritmo del Test di Miller-Rabin
\hrule
\begin{verbatim}
test_Miller-Rabin(N,k):
    for i<- 1 to k:
        y<-rand(2,N)
        if Verifica(N,y):
            return sicuramente-composto


    return forse-primo}
 \end{verbatim} \hrule
La probabilità che sia primo è esattamente $1-\frac{1}{2^k}$\footnote{$1$ meno la probabilità congiunta che sia composto pur avendo superato il test $k$-volte, che è il prodotto delle probabilità semplici se si scelgono $y$ in modo casuale e indipendente.}

\subsection{Applicazione: generare numeri primi grandi}
Un problema comune nelle applicazioni crittografiche è generare numeri primi grandi (per algoritmi che usano funzioni one-way basate sul problema della fattorizzazione). Sfruttando il test di Miller-Rabin possiamo definire un generatore di numeri primi.

La quantità di numeri primi minori di $N$ tende asintoticamente a $N/\log{N}$, quindi in un intorno di raggio $\log{N}$ di $N$ c'è mediamente un numero primo. Definiamo l'algoritmo per generare un numero primo di lunghezza $n$ bit\hrule genera\_Primo($n$,$S$):\\

\indent\indent $N\leftarrow 1S1$\,\footnote{$1S1$ è la concatenazione di stringhe binarie.}\\

\indent\indent \textbf{while} test\_Miller-Rabin($N$,$\log N)==$ composto:\\
\indent\indent\indent $N\leftarrow N+2$\\

\indent\indent \textbf{return} $N$\\

\hrule

\newpage

\section{Cifrari perfetti}

Indicando con $m\in$\textbf{Msg} un messaggio nell'insieme messaggi possibili, con $c\in$\textbf{Critto} un crittogramma nell'insieme dei crittogrammi possibili, se $\mathcal{P}(M=m)$ è la probabilitò che il messaggio spedito sia $m$ e $\mathcal{P}(M=m|C=c)$ la probabilità che il messaggio spedito sia $m$ condizionata dal fatto che il crittogramma spedito $C$ è stato rilevato essere $c$, allora la definizione di Shannon di \textbf{cifrario perfetto} è\\
\\
\fbox{
\parbox{\textwidth}
    {
Un cifrario è \textbf{perfetto} \textit{sse} per ogni messaggio $m\in$\textbf{Msg} e per ogni crittogramma $c\in$\textbf{Critto} vale \[
\mathcal{P}(M=m\,|\,C=c)=\mathcal{P}(M=m)
\]

    }
}\\
\\
dunque la probabilità che un attaccante possa individuare il messaggio non cambia dopo che questi ha intercetta il suo crittogramma.
\subsection{Teorema sul numero delle chiavi in un cifrario pefretto}

\textbf{Il numero delle chiavi in un cifrario perfetto deve essere maggiore o uguale al numero dei messaggi possibili.}

Il teorema si basa sul fatto che se ci fossero meno chiavi che messaggi dato un certo crittogramma, potremmo escludere alcuni messaggi non ottenibili decrittando quel crittogramma\footnote{Esiste solo una decrittazione per ogni chiave, e sono minori del numero dei messaggi possibili.}. Così alteriamo la probabilità condizionata Sdel messaggio, e il cifrario non sarebbe perfetto (in particolare la probabilità condizionata sarebbe $0$ mentre quella a priori non nulla, invece dovrebbero coincidere per definizione di cifrario perfetto).\\
\hrule
\ \\
\textit{Dimostrazione}: Per ogni messaggio possibile $m$ vale $\mathcal{P}(M=m)>0$. Sia $N_m$ il numero dei messaggi possibili, e sia $N_k$ il numero delle chiavi.
 
 Per assurdo $N_k<N_m$. A un crittogramma $c$ corrispondono $s\le N_k$ messaggi\footnote{Non per forza distinti.} ottenuti provando a decrittare $c$ con tutte le chiavi. Poiché $s\le N_k < N_m$ allora esiste almeno un messaggio $m^*$ impossibile da ottenere da $c$\footnote{La funzione decrittazione non è suriettiva}, dunque $\mathcal{P}(M=m^*\,|\,C=c)=0\ne \mathcal{P}(M=m^*)>0$ e il cifrario non sarebbe perfetto.
 
 Dunque il numero delle chiavi deve essere maggiore o uguale al numero dei messaggi se il cifrario è perfetto.
 
 \subsection{Esempio di cifrario perfetto: One-Time Pad}
 
 Il \textbf{one-time pad} è un cifrario perfetto di tipo \textbf{xor}, in cui la chiave è una stringa binaria casuale lunga quanto il testo da cifrare.
 La cifratura avviene bit-a-bit come $c_i\leftarrow m_i\oplus k_i$ dove $c_i,m_i,k_i$ sono rispettivamente lo $i$-esimo bit di crittogramma, messaggio e chiave. La decrittazione avviene simmetricamente con la chiave $k$ bit-a-bit $m_i\leftarrow c_i\oplus k_i$. 
 
 \subsubsection{Teorema: il one-time pad è perfetto}
 Nell'ipotesi che ogni messaggio è possibile e che tutti i messaggi abbiano la stessa lunghezza (basta aggiungere un padding di $0$ per esempio):
 
 Calcoliamo la probabilità condizionata, ottenendo \[
 \mathcal{P}(M=m\,|\,C=c)=\frac{\mathcal{P}(M=m,\,C=c)}{\mathcal{P}(M=m)}=\frac{\mathcal{P}(M=m)\mathcal{P}(C=c)}{\mathcal{P}(M=m)}=\mathcal{P}(M=m)
 \]
 dalla definizione di probabilità congiunta di due eventi indipendenti $\mathcal{P}(M=m,\,C=c)=\mathcal{P}(M=m)\mathcal{P}(C=c)$ e dalla definizione di probabilità condizionale (al secondo membro dell'uguaglianza).
 
 \textbf{NOTA BENE}\\
 Anche nel caso dei linguaggi naturali in cui non tutte le stringhe di $n$ bit sono sensate e appartengono al linguaggio, il one-time pad rimane perfetto. Il numero minimo di chiavi (cardinalità dello spazio delle chiavi) richiesto è inferiore ma sempre esponenziale nella lunghezza del messaggio, dunque il one-time pad in questo caso non ha un numero di chiavi \textit{minimale}.
 
 
\section{Euclide Esteso}
Dati due numeri $a,b$ cerchiamo la tripla di interi $(MCD(a,b),x,y)$ dove $ax+by=MCD(a,b)$:
\hrule
Euclide\_Esteso($a,b$):\\
\indent\indent \textbf{if} $b==0$://caso base\\
\indent\indent\indent \textbf{return} $(a,1,0)$\\
\indent\indent (mcd,$x,y$) $\leftarrow$ Euclide\_Esteso($b,a\mod b$)\\//\^\, scambio $a$ con $b$ e $b$ con il modolo di $a$\\
\indent\indent \textbf{return} mcd,$y,x-\lfloor\frac{a}{b}\rfloor y$ \footnote{$\lfloor\frac{a}{b}\rfloor$ è il quoziente intero della divisione di $\frac{a}{b}$.}
\hrule

\section{Cifrario a chiave pubblica}
Per $n$ utenti bastano $n$ coppie di chiavi pubblica-privata (contro le circa $n^2$ dei cifrari simmetrici\footnote{$n(n-1)/2$}).

Chiavi $k[pub]$ per cifrare verso un destinatario e $k[prv]$ usata dal destinatario per decifrare. La $k[pub]$ è conosciuta da tutti, la $k[prv]$ solo dal destinatario.\\
\textit{Mitt} invia a \textit{Dest} il messaggio $m$ come $C(m,k[pub])=c$ \# computazionalmente facile\\
Dest decritta $c$ con $D(c,k[prv])$ \# anche questo facile\\
L'eventuale funzione $D(c,k[pub])$ che un \textit{attaccante} potrebbe sfruttare è invece difficile computazionalmente.

Questi cifrari sono tuttavia più lenti dei simmetrici (in cui $k[pub]=k[prv]$).

Sono esposti ad audit chosen-plain-text: se voglio sapere se un messaggio specifico $m*$ viene inviato a un certo Dest mi basta calcolare $C(m*,k[pub])$ e controllo se questo crittogramma appaia sul canale di comunicazione.

\section{RSA}
RSA:\\
\`E un cifrario a blocchi a chiave pubblica basato sulla funzione \textit{one-way} esponenziale modulare.\\
\indent\indent k[pub]=($e$,$n$) con $n=pq$ e $p$ e $q$ primi e $e<\Phi(n)$\footnote{Funzione Totiente di Eulero: numero di primi minori dell'argomento. Nel caso del prodotto di due primi vale $(p-1)(q-1)$, altrimenti $\prod 1-\frac{1}{p}$ dove $p$ è un fattore primo di $n$.} 

\indent\indent k[prv]= $e^{-1}_\Phi$, l'inverso di $e$ modulo $\Phi(n)$\\
\textit{Cifratura:} Divido il messaggio in blocchi $m$ di $\lfloor \log_2 n\rfloor$ e applico\\
\indent\indent\indent\indent\, C(m,k[pub])$=m^e\mod n$\\
\textit{Decifrazione:} D(c,k[prv])=$c^{e^{-1}_\Phi}\mod n$

Algoritmo di generazione delle chiavi:
\hrule
\begin{enumerate}
    \item[0.] Scelgo $p,q$ primi di (almeno) un migliaio di bit (per esempio con Miller-Rabin).\\ Calcolo le chiavi in 3 fasi: \\
    \item $n\leftarrow pq$\\
    \indent\indent $\Phi(n)\leftarrow (p-1)(q-1)$
    \item Scelgo un $e<\Phi(n)$, coprimi fra loro\footnote{$e$ chiamato così perché è l'esponente nella cifratura}\\ Se non fossero coprimi la funzione di cifratura non sarebbe invertibile
    \item Calcolo $e^{-1}_{\Phi(n)}$ \# inverso di $e$ $\mod \Phi(n)$, cioè $e\,e^{-1}_{\Phi(n)} \mod \Phi(n)=1$, calcolato con Euclide Esteso
    \item k[pub]$\leftarrow ($e$,$n$)$, k[prv]$\leftarrow e^{-1}_{\Phi(n)}$
\end{enumerate}
\hrule

\subsection{Dimostrazione di correttezza}
Poiché abbiamo scelto $e$ coprimo, dobbiamo dimostrare solo che \[
c^{e^{-1}_\Phi(n)} \mod n = (m^e\mod n)^{e^{-1}_\Phi(n)}\mod n= m
\]
Ci sono due casi:
\begin{enumerate}
    \item[$p$ e $q$ non dividono $m$:] Usando le proprietà dell'esponenziazione modulare, $(m^e\mod n)^{e^{-1}_\Phi(n)}\mod n= m^{e\,e^{-1}_\Phi(n)}\mod n$ ma $e\,e^{-1}_\Phi(n)=1+r\Phi(n)$ per definizione, dunque il precedente è uguale a $m$ per le proprietà delle congruenze per $m<n$.
    
    \item[solo uno fra $p$ e $q$ divide $m$:] Sfrutto il fatto di poter dividere la congruenza $c\mod (pq)$ fra $c\mod p$ e $c\mod q$ e riapplico poi l'argomento precedente
\end{enumerate}
Questo vale  comunque finché cifro e decifro il messaggio in blocchi di $\lfloor \log_2 n\rfloor$ bit (cioé il valore numerico del messaggio rappresentato in bit è minore $n$), altrimenti l'inversione delle congruenze non sarebbe possibile in modo univoco.

\subsection{Attacchi a RSA}
\paragraph{La sicurezza di RSA è equivalente alla fattorizzazione di della chiave pubblica}

Oltre ai chosen-plain-text a cui tutti i cifrari a chiave pubblica sono soggetti \textbf{RSA} può subire alcuni altri attacchi per la sua struttura matematica. La sicurezza del cifrario si fonda sulla difficoltà di fattorizzare $n$ quando è un numero composto molto grande, perché estrarre la radice modulare $e$-esima è un problema altrettanto difficile. Un altro attacco potrebbe essere calcolare $\Phi(n)$ e trovare dalla parte $e$ di k[pub] la chiave privata, ma questo ancora una volta è altrettanto difficile che fattorizzare $n$, dimostrazione:
\hrule
Supponiamo che esista un algoritmo per calcolare $\Phi(n)$ senza scomporre prima $n$ in fattori primi. Tuttavia possiamo calcolare la scomposizione in fattori di $n$ da $\Phi(n)$ in tempo polinomiale, quindi questo problema è computazionalmente equivalente alla scomposizione, infatti: $Phi(n)=(p-1)(q-1)=pq-(p+q)+1=n-(p+q)+1$ dunque $p+q=n-\Phi(n)+1$ e $(p-q)^2=(p+q)^2-4n$ da cui possiamo calcolare $p$ e $q$ in tempo polinomiale.
\paragraph{Accorgimenti per i casi vulerabili}

Se si sceglie la chiave privata nel modo sbagliato rompere il \textbf{RSA} diventa piuttosto semplice. Oltre a richiedere numeri lunghi almeno qualche migliaio di bit, a causa dell'efficienza corrente dell'aritmetica hardware, bisogna scegliere $p$ e $q$ distanti fra loro: se $|p-q|$ fosse piccolo $\frac{p+q}{2}$ sarebbe vicino a $\sqrt{n}$, e dal valore di questa radice basterebbe cercare un valore $z$ in $[\sqrt{n},n)$ tale che $z^2-n$ è un quadrato perfetto, e testarlo eventualmente sul crittogramma. Nella pratica funziona molto velocemente.

Altri attacchi più sofisticati sono possibili se il MCD di $p-1$ e $q-1$ (che sono sicuramente composti perché $p,q$ sono primi) è grande.


\hrule


Esistono vincoli di sicurezza anche sulla scelta dell'esponente $e$, per esempio se questo non induca alcun tipo di cifratura sul messaggio (per $e=1+\Phi(n)/k$ dove $k$ è un divisore di $\Phi(n)$).

Un problema peggiore è dato dal \textbf{teorema cinese del resto}, se più utenti scelgono uno stesso $e$ piccolo: si passa così da dover forzare il cifrario con la radice $e$-esima modulare a fare semplicemente la radice $e$-esima aritmetica.

Possibilmente anche $n$ deve essere diverso fra tutti gli utenti per evitare attacchi basati sull'applicazione dell'algoritmo di Euclide Esteso.

Inoltre RSA è a blocchi, per cui per eliminare le sue periodicità possibili è necessario usare un \textit{Cifrario a composizione di Blocchi}.


\section{Diffie-Hellman classico}
\paragraph{Scambio di chiavi di sessione per cifrari ibridi}
Diffie-Hellman è un protocollo di scambio \textit{chiavi di sessione}. Scambiate le chiavi, con metodi a chiave \textit{pubblica}, il resto della conversazione continuerà con un cifrario simmetrico di chiave pari alla chiave di sessione. 
\`E basato sul logaritmo discreto (calcolare potenze modulari è facile, $ \mathcal{O}(\log k) $ moltiplicazioni, tramite il metodo iterativo delle quadrature successive), trovare l'esponente in $y=a^e mod n$ è difficile sapendo $y,n$ o $a$
\begin{enumerate}
    \item k[pub]$\leftarrow (p,g)$\\
    accordo pubblico su $p$ primo grande e $g$ generatore di $Z^*_p$ (che esiste perché $p$ è primo)
    
    \item[2a.] Alice sceglie $k_A[prv]\leftarrow x<p$, invia $k_A[pub]\leftarrow g^x \mod p\le p-1$
    
    \item[2b.] Bob sceglie $k_B[prv]\leftarrow y<p$, invia $k_B[pub]\leftarrow g^y \mod p$
    
    \item[3a.] Alice calcola $k_A[prv]\leftarrow k_B[pub]^{k_A[prv]} \mod p$
    \item[3b.] Bob calcola $k_B[prv]\leftarrow k_A[pub]^{k_B[prv]}\mod p$
\end{enumerate}
ma $k_A[prv]=k_B[prv]=k[prv]=g^(xy)\mod p$, dunque il protocollo è corretto.\\
\textbf{Attacchi:} attacco attivo \textit{man-in-the-middle}, \textit{Eve} si frappone fra Alice e Bob, invia false chiavi e si finge il destinatario con ognuno. Una \textit{Certification Authority} può mitigare il problema.

\section{Crittografia su curve ellittiche e logaritmo discreto}
\subsection{Generalità su curve ellittiche}
\textit{Consigliato}\\
La \textit{Crittografia su curve ellittiche} \textbf{ECC} sfrutta la difficoltà dell'inversione di alcune operazioni semplici sulle strutture algebriche definite sui punti di particolari curve dette ellittiche (per ragioni storiche).

Dato un campo $K$, definiamo \textit{caratteristica del campo} il minimo numero di volte per cui sommando l'elemento neutro moltiplicativo ($1$) con se stesso si ottiene l'elemento neutro additivo ($0$) del campo. Prendendo curve su campi di caratteristica maggiore di 3, considereremo curve ellittiche (sempre esprimibili) nella \emph{forma normale di Weiestrass} \[
y^2=x^3+ax+b
\]
In particolare possiamo definire curve ellittiche sul campo degli interi modulo $p$ $\mathbb{Z}_p$ con $p$ numero primo (maggiore di $3$ per avere la caratteristica del campo compatibile con la forma normale), dunque avremo una \textit{curva ellittica prima} di primo $p$ e coefficienti $a,b$
\[
E_p(a,b) = \{(x,y)|\, y^2 \mod{p} = x^3+ax+b \mod{p}\}
\]
(simmetrica rispetto a $y=\frac{p}{2}$, \textit{non importante}), e definiremo l'\textbf{ordine} della curva come il suo numero di punti, per il quale vale il seguente \textit{teorema di \textbf{Hasse}} (senza dimostrazione): \textit{l'ordine $N$ di una $E_p(a,b)$ verifica $|N-(p+1)|\le 2\sqrt{p}$}.\\
\hrule
\paragraph{Somma su curve ellittiche e LOGARITMO DISCRETO ELLITTICO}
La somma fra punti di una curva ellittica, definita in un modo particolare, forma un gruppo abeliano\footnote{Si comporta come la somma fra numeri interi.}.

\`E possibile inoltre definire dalla somma il prodotto $kP$ di un punto $P$ per un intero $k$ ricorsivamente come $0P=O,\,(k+1)P=P+kP$.

Un punto ha \textbf{ordine} $n$ se è possibile generare $n$ diversi punti della curva ellittica sommando quel punto a se stesso un numero finito di volte (dunque l'\textit{ordine} di qualsiasi punto è minore all'\textit{ordine della curva}, cioè al suo numero totale di punti).

Così siamo in grado definire l'analogo ellittico del logaritmo discreto: dati due punti $Q,P$ trovare il minimo $k$ tale che $Q=kP$, ossia il logaritmo discreto in base $P$ di $Q$ $\log_P Q$ è il minimo intero verificante
\[
Q=\log_P(Q)\, P
\]

e la sua esistenza è quindi intimamente connessa all'ordine di $P$.

\paragraph{Raddoppi successivi e complesssità} Algoritmo di $\mathcal{O}(\log k)$ prodotti per calcolare $kP$ con i raddoppi successivi:
 binary($k,i$) è l'$i$-esimo bit dal meno significativo al più significativo della rappresentazione binaria di k
\hrule
RaddoppiSuccesivi$(P,k)$:\\
\indent\indent $tmp \leftarrow P$\\
\indent\indent $sum \leftarrow \text{binary}(k,0)\, tmp$\\
\indent\indent \textbf{for} $i\leftarrow 1$ \textbf{to} $\lfloor \log_2 k\rfloor$\\
\indent\indent\indent\indent $tmp\leftarrow 2\,tmp$\\
\indent\indent\indent\indent $sum \leftarrow \text{binary}(k,i)\,tmp$\\
/*  sommo solo quando l'$i$-esimo bit di $k$ non è nullo */

\indent\indent \textbf{return} $sum$\\

\hrule

Vi è una identificazione informale fra la difficoltà dei problemi di aritmetica modulare e su curve ellittiche: ai prodotti modulari corrispondono le somme su curve ellittiche, e alle potenze modulari i prodotti su curve ellittiche (con i rispettivi inversi, come detto per il \textit{logaritmo discreto}).

\section{Diffie-Hellman ellittico}

Nella variante ellittica, Alice e Bob si scambiano punti su una curva ellittica. Alla fine questi punti verranno interpretati come un numero tipicamente prendendo l'ascissa $x_P$ del punto e facendone il modulo per $256$: blocco  $\leftarrow x_p\mod{256}$.

k[pub]: $E_p, B\in E_p$ \\
Il punto $B$ si comporta come il generatore in \textit{DH classico}
Se $N$ è l'ordine di $E_p$, allora poniamo $n<N$ l'ordine del punto $B$\\

Alice:\\
\indent\indent k$_A$[prv]$\leftarrow n_A<n$\\
\indent\indent k$_A$[pub]$\leftarrow n_A B$

Bob:\\
\indent\indent k$_B$[prv] $\leftarrow n_B < n$\\
\indent\indent k$_B$[pub] $\leftarrow n_B B$\\

Chiave privata k[prv]$\leftarrow S = n_A n_B B = n_A k_B [pub]= n_B k_A[pub]$\\
Nel protocollo Alice e Bob si scambiano le chiavi pubbliche e calcolano successivamente la chiave privata di sessione come sopra. Come il \textbf{DH} classico è soggetto ad attacchi attivi (tipo \textit{man-in-the-middle}).

\section{Algoritmo di Koblitz e protocollo di El Gamal}
\subsection{Algoritmo di Koblitz}
Non è nota una funzione $f\in P$ per associare un ascissa (per esempio la codifica binaria di un messaggio) a un punto di una $E_p$, necessario per tradurre efficientemente\footnote{Altrimenti si perde la semplicità di cifratura.} problemi numerici in problemi su punti di curve ellittiche. Esiste però un algoritmo randomizzato in tempo polinomiale $RP$ per questo problema conosciuto come \textit{Algoritmo di Koblitz}.

La probabilità $prob=p(P_m\in E_p(a,b))$ di trovare un punto di una curva $E_p(a,b)$ che abbia un certo valore $m$ come ascissa è la probabilità che esista un $y$ il cui quadrato sia uguale a $m^3+am+b (\mod{p})$ (in tal caso il polinomio nella $m$ è chiamato \textit{residuo quadratico}). 

Si può dimostrare che $prob = \frac{1}{2}$ (dal teorema di Hasse $|N-(p+1)|\le 2\sqrt{p}|$, \textit{senza dimostrazione})

Dunque prendo (aleatoriamente) un $h$ t.c. $(m+1)h=mh+h<p$\\
\hrule
%\fbox{
%\parbox{\textwidth}{
%\begin{minipage}{\textwidth}
/*scorro le ascisse in $[mh,mh+h)$ finché non trovo una associata a $P\in E_p$*/
\textbf{for} $i\in[0,h)$:\\
\indent\indent $x\leftarrow mh + i$\\

\indent\indent \textbf{if} $x^3+ax+b (\mod p)$ è un \textit{residuo quadratico} \\
\indent\indent\indent\indent \textbf{return} $x$\\
 %\end{minipage}
 %   }
%}
\hrule
Probabilità di successo $1-2^{-h}$ (tende a $1$ esponenzialmente)\footnote{La probabilità di trovare ogni volta numeri che non siano residui quadratici, se $h$ è casuale, è il prodotto delle probabilità a priori che non lo sia, che per il teorema citato è $0.5$, dunque $0.5^{numero\ tentativi}$. Poi è stata presa la probabilità complementare.}.

Decrittazione $P_m=(x,y)\to m = \lfloor\frac{x}{h}\rfloor$ dove $m$ è la codifica numerica del messaggio.

\subsection{El Gamal}
Un protocollo crittografico che sfrutta algoritmi simili al precedente è per esempio il cifrario asimmetrico di \textit{El Gamal}

k[pub] = ($E_p(a,b),B$) con $B$ di ordine $n$ elevato

Scelta chiave: $n_{user} < n$, $P_{user} = n_{user}B$

Cifratura:
\begin{enumerate}
    \item $m\mapsto P_m$ con per esempio \textit{Koblitz}
    
    \item prendo a caso $r$: $V\leftarrow rB,W\leftarrow P_m+rP_{dest}$ dove $P_{dest}$ è chiave pubblica del destinatario
\end{enumerate}
Decifrazione: con $n_D$ la chiave privata in mano al destinatario,  ricordando $P_D=n_DB$
\[
P_m = W-n_D V=P_m+rP_D-n_DrB
\]
Se un crittoanalista intercetta $(V=rB,W=P_m+rP_D)$ dovrebbe calcolare $r$ per conoscere il messaggio, dunque risolvere un problema di logaritmo discreto su $EC$.

\section{Sicurezza su curve ellittiche vs altri}
Per il problema della \textit{fattorizzazione} e per il \textit{logaritmo discreto modulare} esistono algoritmi \textbf{subesponenziali}\footnote{\textit{index calculus}, $\mathcal{O}(2^{\sqrt{n\log n}}$ contro lo $\mathcal{O}(2^n)$ di bruteforce}.
Per il \textit{logaritmo discreto} \textbf{su curve ellittiche} il miglior algoritmo è invece ancora pienamente esponenziale\footnote{Al 2014 il \textit{Pollard} $\rho\in \mathcal{O}(2^{\frac{n}{2}})$ che è ancora pienamente esponenziale perché $2^{\frac{n}{2}}=\sqrt{2}^n$.}.

Inoltre i problemi su \textbf{curve ellittiche}, a parità di una complessità dell'implementazione con \textbf{RSA}, permettono chiavi nell'ordine di centinaia di bit, contro le diverse migliaia dei cifrari asimmetrici non basati sulle curve ellittiche per sicurezza simile. 

Possiamo rifarci, per comparazione a una \textbf{misura} ``fisica'' \textbf{universale di sicurezza}: l'energia\footnote{Quantomeno allo stato attuale dell'arte.} per forzare e.g. \textbf{RSA} a 228 bit è pari circa a quella impiegata per bollire un cucchiaio d'acqua, mentre quella per forzare un cifrario basato sulle \textbf{curve ellittiche} a pari lunghezza di chiave è circa l'energia necessaria a far bollire tutta l'acqua presente sulla Terra.

\section{Zero-knowledge Proof}

%Una peculiarità dei protocolli crittografici asimmetrici risiede in parte nel fatto che un utente non possiede parte delle informazioni necessarie a determinare completamente un cifrario; si ha una \textit{asimmetria della conoscenza dell'informazione}, anche se complessivamente ogni chiave, per esempio, è necessaria al corretto svolgimento della comunicazione sicura. Un'altra peculiarità degli scambi crittografici è ovviamente l'\textit{interazione} fra gli utenti nel corso delle computazioni necessarie alla comunicazione e l'utilizzo di sorgenti o algoritmi randomici nei calcoli.

%Uno step successivo sono le
Nelle \textbf{zero-knowledge proof} le computazioni, in questo caso i protocolli crittografici, sono viste come \textit{dimostrazioni} \textit{interattive} (ossia c'è uno scambio di messaggi durante l'esecuzione) \textit{a conoscenza zero}.
Illustriamo il concetto con un esempio. 

Immaginiamo di avere una deficienza nella visione dei colori\footnote{Quindi non distinguiamo i colori ma solo le forme} e che un normovedente voglia dimostrarci che le due palle che abbiamo in mano, che a noi sembrano perfettamente identiche, siano di diverse. Il normovedente ne vede i colori, che distingue e conosce solo lui. 

Possiamo essere \textbf{\textit{probabilmente}} sicuri, con una probabilità grande a piacere, del fatto che non menta e che trovi delle differenze nelle due palle se il normovedente indica un certo numero consecutivo di volte fissato da noi quale palla era originariamente a sinistra o destra dopo che noi le abbiamo scambiate in segreto da lui: se almeno una volta non indovinasse saremmo sicuri che mentiva, mentre la probabilità in $k$ ripetizioni che indovini quale fosse la palla originariamente a destra o sinistra se le palle sono uguali è $\frac{1}{2^k}$. 

In tutto questo, se effettivamente il normovedente ne riconosce i colori non abbiamo potuto provare che mentiva, verificando di volta in volta la correttezza delle sue indicazioni. Se invece è disonesto potrà ingannarci solo con una probabilità infinitesima. In ogni caso così non saremo venuti a conoscenza dei colori delle sfere.

Una \textbf{zero-knowledge proof} vede due utenti, il provatore $P$ e il verificatore $V$. $P$ cerca di \textit{provare} a $V$ che conosce un certo \textit{segreto} (una informazione) senza rivelarlo a $V$, mentre $V$ \textit{verifica} che gli argomenti di $P$ siano logicamente corretti. Una \textbf{Z-K Proof} ha tre caratteristiche:
\begin{enumerate}
    \item[] \textbf{Completezza}: se una affermazione $q$ di $P$ è vera, allora passerà sempre la verifica di $V$.
    
    \item[] \textbf{Correttezza} (probabilistica, asintotica): se una affermazione $q$ di $P$ è falsa, la probabilità che possa passare la verifica di $V$ è $\le \frac{1}{2^k}$ per un $k$ arbitrario scelto di volta in volta da $V$.
    
    \item[] \textbf{Zero-Knowledge}: se $P$ possiede davvero un segreto $s$, anche se $V$ fosse disonesto\footnote{Facesse qualsiasi azione non standard per tentare di forzare il protocollo.} nell'uso del protocollo sarebbe strutturalemente impossibile per lui ottenere dei bit del segreto di $P$.
\end{enumerate}

Un ottimo esempio di impiego di un sistema simile, per l'affidabilità illimitata e la \textit{conoscenza zero} sono i sistemi di identificazione, classicamente soggetti a diversi tipi di furto delle informazioni specie se nel caso di autorità centrali disoneste. 

Un sistema \textit{zero-knowledge} necessità comunque dell'utilizzo di funzioni \textit{one-way}, rendendo però il protocollo a \textit{conoscenza zero}.

\subsection{Protocollo Fiat-Shamir}
Il protocollo \textit{zero-knowledge} \textit{Fiat-Shamir} ha come principale obiettivo l'\textit{autenticazione} (verifica dell'identità di un utente) senza che vi siano \textit{Certification Authority} e senza comunicare alcun bit dei propri dati segreti (chiavi private). Questo permette di risolvere problemi di sicurezza insiti in molti protocolli di autenticazione. 

Utilizza la funzione one-way $s^2\mod n$ (dove $n$ è prodotto di due primi grandi), perché calcolare $s^2 \mod n$ è semplice mentre l'estrazione della radice (in questo caso quadrata) modulare è difficile\footnote{Confrontare con \textbf{RSA} dove il problema difficile era l'estrazione di una generica radice $n$-esima. Non facendo trapelare bit sulla chiave segreta in \textit{Fiat-Shamir} evitiamo i calssici problemi di \textbf{RSA}.}.

Sia $n=pq$ con $p$ e $q$ primi
Sia $s<n$
k[\textit{prv}]= $s$
k[\textit{pub}]=$(n,s^2\mod n = S)$

$V$ (verificatore) conosce la chiave pubblica.

Il protocollo consiste in un numero arbitrario di iterazioni deciso da $V$ che si svolgono come la seguente interazione:\\

Inizio interazione\\
$P$ sceglie un $r<n$ poi invia ${r^2\mod n = R}$ a $ V$\\
$V$ risponde con un bit casuale $e\in \{0,1\}$ inviandolo a $P$\\
$P$ deve ora calcolare $r s^e \mod n$ e inviarlo a $V$\\
fine interazione\\

$V$ può ora verificare\footnote{Con un errore $\frac{1}{2}$, che tenderà a $0$ aumentando le iterazioni} che $P$ sia in possesso del valore $s$ conoscendo solo k[\textit{pub}] facendo:\\
\hrule
\ \\
\textbf{if} $R S \mod n == (r s^e \mod n)^2 \mod n$\\
\indent\indent \textbf{return OK}\\
\textbf{else return} \textit{non verificato}\\
\ \\
\hrule
\ \\

La \textbf{\textit{completezza}} del protocollo discende dal fatto che se $P$ conosce davvero $s$ (cioè la radice quadrata modulare di $S$ della chiave pubblica), qualunque sia $r$ scelto all'inizio e qualunque sia il bit $e$ inviato da $V$, $R S \mod n = (r^2 \mod n)(s^2\mod n)\mod n= (rs)^2\mod n$\ \,\footnote{Dalla proprietà dell'aritmetica modulare $ab\mod n=(a\mod n b\mod n)\mod n$ e dalle proprietà delle potenze.} e sarà dunque uguale a $ (r s^e \mod n)^2 \mod n$



\end{document}
